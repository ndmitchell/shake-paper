\section{Challenges of large-scale build systems}

Summary of this section: \emph{What problem are we trying to solve?} We show several
common challenges for existing build languages, such as \textsc{Make},
\textsc{cmake}, \textsc{Ninja} and others (the list to be updated). Each
challenge is described in a separate section, with specific examples coming from
the GHC. We briefly outline the solutions and refer the reader for more details
coming in Sections~\ref{section-abstractions} and~\ref{section-dependencies}.

\subsection{Abstracting over patterns of dependencies}

Weak abstractions of \textsc{Make}:
\begin{itemize}
  \item Pattern rules: do not work in general
  \item Macros: untyped, Utterly Impenetrable 
\end{itemize}

\noindent Consider the following pattern rule:

\begin{verbatim}
%.o : %.hs
    ghc HC_OPTS $<
\end{verbatim}

\noindent It tells \textsc{Make} that object files \texttt{*.o} can be produced
from Haskell source files \texttt{*.hs} by compiling them with \texttt{ghc}
command invoked with \texttt{HC\_OPTS} arguments. The notation is terse and
works well for this simple case, but as soon as one goes beyond \textit{one rule
fits all} the following limitations are encountered:
\begin{itemize}
  \item What if we want the rule to match \texttt{foo.o} and \texttt{bar.o}, but
  not \texttt{baz.o}? It is impossible to do any non-trivial computation and
  we have to rely on patterns whose expressive power is limited.
  \item What if \texttt{HC\_OPTS} depends on the file and/or something else?
  \item GHC actually produces two files: \texttt{*.o} and \texttt{*.hi}. How do
  we express this?
\end{itemize}

The standard approach to overcome some of \textsc{Make}'s limitations is to use
\emph{macros}. They provide more flexibility but tend to become Utterly
Impenetrable as demonstrated by the following \emph{significantly simplified}
snippet from the GHC build system:

\begin{verbatim}
$1/$2/build/%.o : $1/$4/%.hs $$$$($1_$2_HC)
    $1_$2_HC $$($1_$2_$3_HC_OPTS) -c $$< -o $$@
\end{verbatim}

\noindent As before, the rule is responsible for compiling a Haskell source
file into an object file. Arguments of the macro (\texttt{\$1-\$4}) provide
additional information about the current build target so that we could pick the
right Haskell compiler \texttt{\$1\_\$2\_HC} and run it with appropriate command
line arguments \texttt{\$1\_\$2\_\$3\_HC\_OPTS}.

By embedding the build system in Haskell we get access to its rich abstraction
facilities. See Section~\ref{section-abstractions} for more details.

\subsection{Programming model}

\textsc{Make}'s programming model: single global space of string variables. Also
see: ``Recursive Make Considered Harmful'' by Peter Miller, 1997. Specific problems:
\begin{itemize}
  \item A string variable, such as \texttt{\$1\_\$2\_\$3\_HC\_OPTS}, can be
  manipulated in many different files and it is very difficult to track the
  provenance of a specific command line argument that is eventually passed to
  \texttt{ghc}.
  \item Lack of encapsulation, implementation hiding, namespaces make it
  impossible to write maintainable code.
  \item If a file's path contains `:' character, which often happens on Windows,
  \textsc{Make} interprets it as a pattern rule separator and gets confused.
  Special care needs to be taken to overcome this issue.
\end{itemize}

See Section~\ref{section-abstractions} discussing our approach to solving
these problems.

\subsection{Controlling concurrency}
The only way to control build concurrency in \textsc{Make} is by introducing
fake \emph{concurrency reduction} dependencies. 

\textbf{Example:} Some GHC packages need to be registered by invoking
\texttt{ghc-pkg} utility. The utility mutates the global state (package
database) and hence at most one package can be registered at a time. In order to
enforce sequential package registration in the old build system it is necessary
to manually introduce fake dependencies between packages. Not only this is an
an error prone and mundane task, the resulting performance is suboptimal: only
one predefined registration sequence is allowed and packages are required to
wait for their turn even if no registration is currently taking place.
\textsc{Shake} provides fine-grain mechanisms for controlling concurrency, as
discussed in Section~\ref{section-abstractions}. 

\subsection{Dynamic dependencies}

\textsc{Make} has no support for \emph{dynamic dependencies}. Build
phases do not work in general. Below is a representative dependency chain:

\begin{itemize}
  \item File \texttt{compiler/prelude/primops.txt.pp} describes GHC’s primitive
  operations and types. It uses C preprocessor directives like
  \texttt{\#include} and \texttt{\#if} and therefore needs to be preprocessed.
  The result goes into \texttt{compiler/stage1/build/primops.txt}. If one of the
  included files changes the result must be rebuilt. Note: we do not know in
  advance which files are included, so we cannot depend on them statically,
  that is why such dependencies are called dynamic.
  \item File \texttt{primops.txt} is processed by \texttt{genprimopcode} utility
  to generate \texttt{primop-out-of-line.hs-incl}. \texttt{genprimopcode} itself
  needs to be built. It is a Haskell program, so it may contain \texttt{import}
  directives, which also give rise to dynamic dependencies.
  \item A dozen of \texttt{*.hs-incl} files are included (by a C preprocessor)
  into \texttt{compiler/prelude/PrimOp.hs} and also need to be tracked
  dynamically.
  \item \texttt{compiler/prelude/PrimOp.hs} is imported in many other source
  files, and participates in the rest of the build process in a more or less
  standard way.
\end{itemize}

\noindent Creating a separate build phase for each step in the above chain is
impractical and leads to poor performance results.

We handle dynamic dependencies using \textsc{Shake}'s oracles. See
Section~\ref{section-dependencies} for more details.

\subsection{Fine-grain dependencies}

\textsc{Make} supports only file dependencies, so it is impossible to depend
on anything finer, e.g. a single flag in a configuration file. Keeping flags
separately in different files is impractical. 

See Section~\ref{section-dependencies} for more details.
