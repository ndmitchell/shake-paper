\documentclass[preprint]{sigplanconf}

\usepackage{amsmath}
\usepackage{amssymb}
\usepackage{natbib}
\usepackage{multirow}
\usepackage{setspace}
\usepackage{balance}
\usepackage{verbatim}
\usepackage[T1]{fontenc}  % access \textquotedbl
\usepackage{textcomp}
\usepackage[all]{xy}
\include{paper}

\lstset{
    basicstyle=\sffamily,
    columns=fullflexible,
    keepspaces=true,
    escapechar=\#,
    literate={-}{{\hbox{-}}}1
             {_}{{\hbox{\tiny\_}}}1
             {"}{{\textquotedbl}}1
             {->}{{$\rightarrow$}}1 {<-}{{$\leftarrow$}}1
             {=>}{{$\Rightarrow$}}1
             {>>}{{>\!>}}1
             {<<}{{<\!<}}1
             {<>}{{<\hspace{-0.4mm}>}}1
             {</>}{{<\hspace{-0.4mm}/\hspace{-0.4mm}>}}1
             {<.>}{{<\hspace{-0.3mm}.\hspace{-0.3mm}>}}1
             {<\$>}{{<\hspace{-0.3mm}\$\hspace{-0.3mm}>}}1
             {<*>}{{<\hspace{-0.3mm}*\hspace{-0.3mm}>}}1
             {&&&}{{$\wedge$}}1
             {|||}{{$\vee$}}1
             {!!!}{{$\neg$}}1
             {===}{{$\equiv$}}1 }
\newcommand\lst\lstinline

\newcommand\prog\textsc
\newcommand{\make}{\prog{Make}}
\newcommand{\cabal}{\prog{Cabal}}

\begin{document}

\setlength{\pdfpageheight}{\paperheight}
\setlength{\pdfpagewidth}{\paperwidth}

\conferenceinfo{ICFP'16}{September 18--24, 2016, Nara, Japan}
\copyrightyear{2016}
%\copyrightdata{978-1-nnnn-nnnn-n/yy/mm}
%\copyrightdoi{nnnnnnn.nnnnnnn}
% Uncomment the publication rights you want to use.
%\publicationrights{transferred}
%\publicationrights{licensed}     % this is the default
%\publicationrights{author-pays}
%\titlebanner{banner above paper title}        % These are ignored unless
%\preprintfooter{short description of paper}   % 'preprint' option specified.

\title{Non-recursive Make Considered Harmful}
\subtitle{Build Systems at Scale}

\authorinfo{Authors}
           {Affiliations}
           {Emails}
%\authorinfo{Name2}
%          {Affiliation2}
%          {Email2}

\maketitle

\begin{abstract}
\todo{\textbf{SLPJ}: Rewrite}
The paper discusses common challenges of build systems at scale and proposes a
number of techniques that help overcome them. The techniques have been
implemented in Haskell and thoroughly tested through a complete redesign of an
existing build system of the Glasgow Haskell Compiler.
\end{abstract}

%\category{CR-number}{subcategory}{third-level}
\keywords
build-system, compilation, Haskell

\section{Introduction}

A build system is a mission-critical component of any software project.
It is responsible for collecting all source code written in various languages,
compiling it, linking with standard libraries, running automated tests, and
producing executable artefacts -- end software products. Build systems of large
software projects, such as the Glasgow Haskell Compiler, are complex engineering
artefacts. Their complexity stems from several factors: they evolve everyday as
new features are continuously being added by engineers spread across multiple
continents, they must support a variety of target hardware configurations, and
last but not least they operate under extreme correctness and performance
requirements because they stand on the critical path between the development of
a new feature and its deployment into production.

\section{Challenges of large-scale build systems\label{sec:challenges}}

It's taken 4 iterations to get to:

\begin{lstlisting}[basicstyle=\footnotesize\sffamily,escapeinside={(*}{*)}]
$1/$2/build/%.$$($3_osuf) : $1/$4/%.hs $$(LAX_DEPS_FOLLOW) \
    $$$$($1_$2_HC_DEP) $$($1_$2_PKGDATA_DEP)
  $$(call cmd,$1_$2_HC) $$($1_$2_$3_ALL_HC_OPTS) -c $$< -o \
    $$@ $$(if $$(findstring YES,$$($1_$2_DYNAMIC_TOO)),-dyno \
    $$(addsuffix .$$(dyn_osuf),$$(basename $$@)))
  $$(call ohi(*-*)sanity(*-*)check,$1,$2,$3,$1/$2/build/$$*)
\end{lstlisting}

This shows lots of the problems.


\todo{Shorten the problem descriptions, using the footnote as an overall 
  motivating example. Unfootnote it.}

\begin{enumerate}
  \item jmake - cpp + make clone of X11 imake
  \item GNU make - drop cpp, recursive make, manual stages
  \item GNU make - no manual stages, cabal integration, first
   use of macros
   \item GNU make - current iteration, non recursive, cite Recursive
     Make Considered Harmful here, extensive use of macros,
     building abstractions in make
\end{enumerate}

\subsection{The GHC build system}

Build systems need to evolve quickly with the projects they support. While it is
\textit{possible} to develop large scale systems in \make, it is not
\textit{pleasant} -- resulting in heroic efforts where straight-forward
simplicity should be preferred.

The GHC build system is a complex multi-language build system. The system is a
bootstrapping compiler, where the GHC compiler is built using a system compiler,
then recompiled using built compiler and recompiled system libraries. This
pattern naturally contains repeated patterns, but capturing them in \make{} is
hard.

We focus on the GHC build system for two reasons. Firstly, it is the coal-face
at which many of us work, so improvements to it will bring real improvements to
our daily development. Secondly, the GHC build system has many complex features
that test the limits of existing features:

\begin{itemize}
\item GHC is cross-language, including large amounts of both C and Haskell code.
It also generates user manuals from docbook, which can be viewed as another
language with unique build/dependency patterns.
\item GHC is a bootstrapping compiler with stages. It first builds a compiler
using the system compiler, then uses that new compiler
\item The GHC build system necessarily integrates with other build systems,
including \make{} (for building libgmp) and \cabal{} (for building/registering
Haskell libraries).
\item It generates files, e.g. \texttt{compiler/stage1/ghc\_boot\_platform.h}
that \texttt{\#define}'s various platform-specific constants used throughout the
build system.
\item It is cross-platform, working on Windows, Linux, Mac, iOS, Android,
Solaris, BSD flavours etc.
\item The system is in constant flux as new features are added to the system.
\end{itemize}

Writing such a build system remains a challenging engineering undertaking, but
one we now hope not to have to repeat.

The GHC build system stumbles into a number of nasty corners of \make{}. In this section we reflect on the challenges, and how they can be solved generically in our new system. We very much consider these \textit{unnecessary} complexities -- they are not consequences of our problem domain, merely weaknesses of \make{} on large projects. In general the problems can be divided into those due to the \make{} language (macros, variables etc.) and those due to the \make{} dependency features (lack of expressive dependencies). Consequently, we tackle these problems using functional programming for the language level, and Shake for the dependency level. After tackling these unnecessary complexities, we show how to construct a large build system \S\ref{sec:abstractions}.

\section{Background about Shake\label{sec:shake}}

The Shake build system was introduced by \citet{shake}. In this section we briefly recap the key ideas of Shake from that paper. Next we discuss some of the inovations in Shake since that time. All are generally useful and were in wide use before we started investigating the GHC build system. Much of this section can be considered the bits needed to take us from the theory to the practice.

\begin{figure}
\begin{lstlisting}
newtype Rule a = ... deriving (Monoid, Functor, Applicative, Monad)
newtype Action a = ... deriving (Functor, Applicative, Monad, MonadIO)

shake :: ShakeOptions -> Rules () -> IO ()
action :: Action a -> Rules ()

type ShakeValue a = (Show a, Typeable a, Eq a, Hashable a, Binary a, NFData a)
data EqualCost = EqualCheap | EqualExpensive | NotEqual

class (ShakeValue key, ShakeValue value) => Rule key value where
    storedValue :: Rule key value => ShakeOptions -> key -> IO (Maybe value)
    equalValue :: Rule key value => ShakeOptions -> key -> value -> value -> EqualCost

rule :: Rule key value => (key -> Maybe (Action value)) -> Rules ()
apply :: Rule key value => [key] -> Action [value]
\end{lstlisting}
\todo{Include everything we use in the paper}
\caption{Shake generic API}
\end{figure}

\begin{figure}
\begin{lstlisting}
(?>) :: (FilePath -> Bool) -> (FilePath -> Action ()) -> Rules ()
want :: [FilePath] -> Rules ()
need :: [FilePath] -> Action ()
needed :: [FilePath] -> Action ()
\end{lstlisting}
\caption{Shake file-specific API}
\end{figure}

\subsection{Introduction}

The two key types in Shake are \lst"Rules" and \lst"Action". The \lst"Rules" represents the list of things Shake knows how to build, and each rule has an associated \lst"Action" which is the list of actions to build. The \lst"Rules" is a monoid, allowing two sets of rules to be joined to form a new set of rules. The action is a \lst"MonadIO" allowing actual actions to be run. As an example:

\begin{lstlisting}
main :: IO ()
main = shake shakeOptions $ do
    want ["foo.o"]

    "*.o" %> \out -> do
        let src = out -<.> "c"
        need [src]
        cmd "gcc -o" out "-c" src
\end{lstlisting}

There the whole block is a \lst"Rules", while the 5 indented lines are the \lst"Action". The action says what to match, namely all files with a \lst".o" extension - object files. The action gets run with \lst"out" bound to the output name, e.g. \lst"Foo.o". It then computes \lst"src" (e.g. \lst"Foo.c") and \lst"mk" (e.g. \lst"Foo.m"). It requires the \lst".c" file to exist and builds it if it is missing, then compiles, and finally introduces a dependency on all the files listed in the \lst".m" file.

This very simple build system compiles a single file. The \lst"want" says what we want to be available after the build has completed. The \lst"*.o" pattern says it can build any \lst".o" file. The \lst"need" says that the \lst"src" file must be built before continuing. The \lst"cmd" is a poly arity function which takes and executes command lines \cite{poly_arity}.

Unlike Make, Shake has a separate database. If the output of the rule (foo.o) or any of the inputs (in this case just foo.c) change then the rule will rebuild. Shake stores information about each file in a separate database. Note that any change of the file modification time will cause a rebuild, not merely if the output is older than the inputs.

Looking at the body of the rule, we have tracked what happens if the .c file changes, but what if any headers that the .c file includes change? The program gcc takes a -M option that lets us save the dependencies to a file:

\begin{lstlisting}
let src = x -<.> "c"
need [src]
Stdout mk <- cmd "gcc -o" out "-M" "-c" src
need $ makefileDepends mk
\end{lstlisting}

We rely on an auxiliary \lst"makefileDepends" which parses a Makefile and returns all the dependencies \footnote{Using the \lst"parseMakefile" function in Shake we can define \lst"makefileDepends = concatMap snd . parseMakefile".}. After finding all the dependencies we \lst"need" them, thus ensuring that if a header changes it will be rebuilt.

In fact, since Shake is a monadic build system, the \lst"need" will actually build them - a very useful feature and one that sets Shake apart. However, in this case if the header files are built after the execution of \lst"gcc" then the object file will be incorrect, so instead we can switch the final \lst"need" for \lst"needed". This combines a \lst"need" which an assertion that the file does not change as a result of building.

\subsection{Polymorphic dependencies}

While Shake can depend on files, it can also depend on other things, specifically Fig 1 is all about the core of Shake, while Fig 2 is about a wrapping for files. For example, we can define a rule that produces the version of gcc, allowing wrapping it easily.

\begin{lstlisting}
newtype GccVersion () deriving ...
instance Rule GccVersion String where
    storedValue _ _ = return Nothing
gccVersion = rule $ \(GccVersion _) -> Just $ do
    Stdout s <- cmd "gcc --version"
    return s
\end{lstlisting}

Now we can agument our rule by including:

\begin{lstlisting}
apply [GccVersion ()] :: Action [String]
\end{lstlisting}

Now, if the gcc version changes our rule will rebuild. To ensure this happens when required, our build system will necessarily run the gcc version command in each iteration. To simplify this pattern we wrap the common pattern up as an oracle:

\begin{lstlisting}
gccVersion <- addOracle $ \(GccVersion _) ->
    Stdout s <- cmd "gcc --version"
    return s

gccVersion $ GccVersion ()
\end{lstlisting}

Now we are freed from defining our own Rule instance and have more type safe sugar. The definition corresponds to the pieces above abstracted out and is only a handful of lines.

We do not believe that polymorphic dependencies give any fundamental additional expressive power. What they do is allow greater composition by not having to invent a file name for each thing, by avoiding so many files, and by having far richer keys than simply filenames. We use oracles pervasively in Shake.

If an oracle does not change it does not rerun. This approximately captures the Make pattern of running an action and not updating the output file if it has not changed.

We use oracles pervasively in the GHC build system, fwd ref.

\subsection{Shared cache}

The abstraction.

\subsection{Order-only dependencies}

One dependency feature missing from the original Shake paper was order-only dependencies. An order-only dependency is one that must be built before continuing, but if the order-only dependency changes this rule does not need to rerun. A legitimate use of such a dependency is that an action might read any one of two files, and after the fact, can report which files it actually depended upon. Usually then a subset of these files will be added as explicit dependencies afterwards.

This pattern can be modelled in Shake using a rule whose key type is \lst"()" -- one that always compares equal, and the user defining a tag for the closure required, so it can be stored in the database. While workable, the pattern is not particularly reusable, so we instead provide a function that directly resets the dependency state.

Is this used? Add an example.

\subsection{Resources}

When you run -j10 (shakeThreads=10) you are asking the build system to limit computation so it uses no more than ten CPU resources at a time. The CPU is certainly a precious resource, but there are other resource limitations a build system may need to obey:

\begin{enumerate}
\item Some APIs are global in nature, if you run two programs that access the Excel API at the same time things start to fail.
\item Many people have large numbers of CPUs, but only one slow rotating hard drive. If you run ten hard-drive thrashing linkers simultaneously the computer is likely to grind to a halt.
\item Some proprietary software requires licenses, a fixed number of which can be purchased and managed using a license manager. As an example, the Kansas Lava team only have access to 48 licenses for modelsim.
\end{enumerate}

I know of three approaches used by other build systems to obey resource constraints:

\begin{enumerate}
\item Limit the number of CPUs to hit your target - for example, the Lava build system could cap the number of CPUs to the number of licenses. People with 24 CPUs might ask the build system to use only 8, so the linkers do not make their machines unusable (and even then, a link heavy rebuild may still harm interactive performance). This solution wastes CPU resources, leaving CPUs that could be building your code idling.
\item Add locks to suspend jobs that are competing for the shared resource. For example any rule using Excel could take the Excel lock, either a mutex/MVar in some build systems, or creating a file to serve as the lock in make based build systems. Locking can be made to work, but is tricky if you have to fake locks using the file system, and still squanders CPU resources - instead of blocking the CPU should be running another rule.
\item Use dependencies in sequence to ensure that the items running in parallel are serialised.
\end{enumerate}

In Shake the Resource type represents a finite resource, which multiple build rules can use. Resource values are created with newResource and used with withResource. As an example, only one set of calls to the Excel API can occur at one time, therefore Excel is a finite resource of quantity 1. You can write:

\begin{lstlisting}
shake shakeOptions{shakeThreads=2} $ do
    want ["a.xls","b.xls"]
    excel <- newResource "Excel" 1
    "*.xls" *> \out ->
        withResource excel 1 $
            system' "excel" [out,...]
\end{lstlisting}

Now we will never run two copies of excel simultaneously. Moreover, it will never block waiting for excel if there are other rules that could be run.

Fwd ref to ghc-pkg.

\subsection{Modification tracking}

Build systems run actions on files, skipping the actions if the files have not changed. An important part of that process involves determining if a file has changed. The Make build system uses modification times to impose an ordering on files, but more modern build systems tend to use the modification time as a proxy for the file contents, where any change indicates the contents have changed (e.g. Shake, Ninja). The alternative approach is to compute a hash/digest of the file contents (e.g. SCons, Redo). As of version 0.13, Shake supports both methods, along with three combinations of them - in this post I'll go through the alternatives, and their advantages/disadvantages.

Modification times rely on the file-system updating a timestamp whenever the file contents are written. Modification time is cheap to query. Saving a file afresh will cause the modification time to change, even if the contents do not - as a result touch causes rebuilds. Unfortunately, working with git branches sometimes modifies a file but leaves it with the same contents, which can result in unnecessary rebuilds. We can view the modification time as a surjective function from modification time to file contents.

File digests are computed from the file contents, and accurately reflect if the file contents have changed. There is a remote risk that the file will change without its digest changing, but unless your build system users are actively hostile attackers, that is unlikely. The disadvantage of digests is that they are expensive to compute, requiring a full scan of the file. In particular, after every rule finishes it must scan the file it just built, and on startup the build system must scan all the files. Scanning all the files can cause empty rebuilds to take minutes.

To get the best of both worlds Shake can store the modification time, file size and hash of the contents. After producing a file all information is stored. When checking, first the modification time is checked, and if it matches, the contents have not changed. If the modification time has changed, and the size has changed, then the contents does not match. Only in the case where the modification time has changed but the size has not do we have to compute the actual hash. If after doing that the contents are equal we store a new modification time, so that future checks will be fast. If the contents have changed then the file will likely be rebuilt, and thus will be written afresh with the new hash and modification time.

While a signficant optimisation over always checking file hashes, for certain large files the computation of a hash can still be quite expensive (although almost always cheaper than producing the file). To reduce that problem, Shake has a mode that only digests for source files that are not written by the build system. Generated files (e.g. compiled binaries) tend to be large (expensive to compute digests) and not edited (rarely end up the same), so a poor candidate for digests. The file size check means this restriction is unlikely to make a difference when checking all files, but may have some limited impact when building.

Describe the git pattern. Remove the description of configuration.

\subsection{Lint checks}

\S? of \cite{shake} postulates a number of invariants, and using the \prog{FSATrace} program these can now be checked at runtime. Shake has a \texttt{--lint} flag which also checks the current working directory does not change (a common mistake in a global build system, as the current working directory is a shared resource). It also checks that files do not change after they have been depended upon, that running a fresh build after a build completes will have no further effect, and that.

We also have a function \lst"needed", rather than \lst"need" that asserts that the result of performing \lst"need" does not cause the file to change. This is typically required to depend on files that have already been used, e.g. the header files have already been scanned, so if they were to build afresh that would be an error.

Most useful one is if two people whack the same output. Other thing is where doing all .hs files in a dir, and then you generate one. Should this go somewhere else? Perhaps in S5.

\subsection{Design pattern: DSL + escape hatch}

\todo{Probably belongs elsewhere, but don't want to conflict}

As we develop more Shake-based build systems, patterns have started to emerge. Typically 90\% of a build system can be captured in some simple DSL, taking advantage of conventions, and 10\% cannot. As an example, a large build system might build 100 C++ libraries, each with similar flags and file layout, but taking the source files from different directories. It may also minify a Javascript file, and build an installer -- both one-off tasks. Using a fully-powerful system such as Shake, it is possible to engineer robust abstractions, and then define the majority of the build system using only these abstractions. The end result is that most edits to the build system involve only the DSL, and can be performed by a large number of individuals.

After dealing with the DSL, there are usually a few pieces left over, and these can be implemented in Shake, as normal. Thanks to the power of Shake you can interpret the DSL and combine it with the custom pieces. In our experience by providing an \textit{escape hatch} where fully-powerful code an be expressed it removes the temptation to shoehorn more advanced features in the DSL, and thus avoids turning it into an ad-hoc scripting language. Instead, should enough pieces be required in the escape hatch, they can be abstracted in the traditional ways - perhaps even combining two DSLs in one build system.

\section{Quick wins: from Make to Shake\label{sec:solutions}}

The GHC build system stumbles into a number of nasty corners of Make. In this
section we reflect on the challenges, and how they can be solved in
our new system. We consider these \empty{unnecessary} complexities --
they are not consequences of our problem domain, merely weaknesses of Make on
large projects. The problems can be divided into those caused by the
Make language (macros, variables, etc.) and those caused by a lack of expressive
power when describing dependencies. We tackle these problems using
functional programming for the language level, and Shake for the dependency
level. After tackling these unnecessary complexities, we show how to structure a
large build system in~\S\ref{sec:abstractions}.

\subsection{Programming model}

Make's program state involves a global namespace of mutable string variables which are
spliced into the program. This model naturally causes challenges:

\begin{itemize}
\item Since variables live in a single global namespace, there is limited
encapsulation and implementation hiding.
\item Since variables are strings, arrays and associatve maps are typically
encoded using computed variable names, which is error-prone.
\item Since variables are mutable, the time at which they are expanded needs
to be carefully controlled, resulting in constructs such as \lst'$$$$foo' to
delay expansion until after interpretation by two macros.
\item Since variable references are spliced into the Makefile contents
and interpretted, certain special characters can cause problems --
notably space (which splits lexemes) and colon (which declares rules, but is
also used in Windows drive letters).
\end{itemize}

The solution to encapsulation is a properly block-scoped language with implementation
hiding, e.g. modules in Haskell. The solution to excessive
freedom is types (not necessarily static). The solution to
splicing is to separate values and program text, much like most programming
languages do. By using Haskell, or indeed any other modern programming language,
most of these issues are eliminated.

An alternative solution is to generate the Makefile, either using a tool such as
Automake or a full programming language. However, a generator cannot interact with the
build system after generation, resulting in problems with dynamic dependecies
(\S\ref{sec:dynamic-deps}).

\subsection{Pattern/rule language\label{sec:pattern-rule-language}}

A build system builds some output files from some input files. The fundamental
unit of work in Make (and also Shake) is a rule that produces some output by
running commands on some input. Consider the following Make pattern rule:

\begin{lstlisting}
%.o : %.hs
    ghc $HC_OPTS $<
\end{lstlisting}

\noindent It tells Make that object files \lst"*.o" can be produced from Haskell
source files \lst"*.hs" by compiling them with \lst"ghc" command invoked with
\lst"HC_OPTS" arguments. The notation is terse and works well for this simple
case. Unfortunately, this simple formulation does not support numerous important
features, for example:

\begin{itemize}
\item What if we want the rule to match \lst"foo.o" and \lst"bar.o", but not
\lst"baz.o"? It is impossible to do any non-trivial computation -- we are forced
to rely on patterns whose expressive power is limited.
\item What if \lst"HC_OPTS" should depends on the file being compiled?
\end{itemize}

The standard approach to overcome some of Make's limitations is to use
\emph{macros}. They provide some additional flexibility, but quickly become
difficult to follow. As an example, here is a rule derived from the example
shown in~\S\ref{sec:challenges}:

\begin{lstlisting}
$1/$2/build/%.o : $1/$4/%.hs $$$$($1_$2_HC)
    $1_$2_HC $$($1_$2_$3_HC_OPTS) -c $$< -o $$@
\end{lstlisting}

\noindent As before, the rule is responsible for compiling a Haskell source
file into an object file. The plethora of dollars works like this:
\begin{itemize}
\item Arguments of the macro are \lst"$1" to \lst"$4".

\item GNU Make has a single global namespace for variables, and yet we
  have a single build system that encompasses many similar components.
  To keep things sane, we have to prefix all our variables with the
  directory name, and because we build multiple instances of things
  (e.g. two stages of the compiler), we have yet more prefixes.  There
  are three prefixes in total, by convention we tend to use \lst"$1",
  \lst"$2", and \lst"$3".

  In this case we pick the right Haskell compiler \lst"$1_$2_HC"
  and run it with command line arguments \lst"$1_$2_$3_HC_OPTS".

\item Expanding a variable before the variable is defined gives an
  empty result.  Thus, if we're defining a macro before the variables
  it refers to, we need to use a double dollar (\lst"$$") to delay the
  expansion until the macro is called.

\item Sometimes we need to delay the
  expansion by yet another step, requiring a double-double-dollar (\lst"$$$$").
  Deciding how many dollars to use in any given place is hard.
\end{itemize}

\noindent Using Shake the simplest variant looks slightly more complex, because
it must be expressed in Haskell syntax:

\begin{lstlisting}
"*.o" %> \out -> do
    let hs = out -<.> "hs"
    need [hs]
    cmd "ghc" hc_opts hs
\end{lstlisting}
\noindent
However, as the complexity grows, so the build system scales properly. Making
\lst"hc_opts" depend on the Haskell file requires the small and obvious change of:

\begin{lstlisting}
    cmd "ghc" (hc_opts hs) hs
\end{lstlisting}
\noindent
To use richer pattern matching we can drop down to a lower level Shake
operation. In Shake the definition of is \lst"%>" is itself defined in terms of
\lst"?>", as:

\begin{verbatim}
pat %> act = (pat ?==) ?> act
\end{verbatim}
\noindent
So the wildcard pattern matching is just a special case, and we can use an
arbitrary predicate to exert more precise control over where the match occurs.
In both cases the generality of higher-order functions solves the problem.

\subsection{Generating rules}

\todo{\textbf{Andrey}: Rewrite.}

In any large build system, there are rules (how to build things), and then a
very long list of corner cases. For example,

\begin{lstlisting}
compiler_ALEX_OPTS = --latin1
\end{lstlisting}

While this is a concise way of representing it, it's inflexible. Rather than
pattern matching on everything, we are pattern matching on two components (which
package we are building and which tool we are running). Adding additional
filters is impossible.

Implementing the DSL is a nightmare. It's really a configuration setting.

The question of provenance is particularly important in large multi-author build
systems, but generally not something that is addressed anywhere - one of our key
inovations in \S?. The encapsulation is provided by Haskell modules. The lack of
freedom is provided by strong name binding. The problems related to `:' go away
by manipulating values rather program text.


As we develop more Shake-based build systems, patterns have started to emerge.
Typically 90\% of a build system can be captured in some simple DSL, taking
advantage of conventions, and 10\% cannot. As an example, a large build system
might build 100 C++ libraries, each with similar flags and file layout, but
taking the source files from different directories. It may also minify a
Javascript file, and build an installer -- both one-off tasks. Using a
fully-powerful system such as Shake, it is possible to engineer robust
abstractions, and then define the majority of the build system using only these
abstractions. The end result is that most edits to the build system involve only
the DSL, and can be performed by a large number of individuals.

After dealing with the DSL, there are usually a few pieces left over, and these
can be implemented in Shake, as normal. Thanks to the power of Shake you can
interpret the DSL and combine it with the custom pieces. In our experience by
providing an \emph{escape hatch} where fully-powerful code an be expressed it
removes the temptation to shoehorn more advanced features in the DSL, and thus
avoids turning it into an ad-hoc scripting language. Instead, should enough
pieces be required in the escape hatch, they can be abstracted in the
traditional ways - perhaps even combining two DSLs in one build system.

\subsection{Rules with multiple outputs\label{sec:multiple-outputs}}

GHC produces object \lst"*.o" and interface \lst"*.hi" files when compiling
Haskell sources. How do we express this in a build rule?

A typical Make appoach to this is to use two rules, one depending on the other.
For example:

\begin{lstlisting}
%.o : %.hs
    ghc $HC_OPTS $<
#\vspace{-2mm}#
%.hi : %.o ;
\end{lstlisting}
\noindent Here \lst';' is a no-op: it tells Make that an interface file can be
produced simply by depending on the corresponding object file. Alas, this
approach is fragile: if one deletes the interface file, but leaves the object
file intact, Make will not rerun the \lst'*.o' rule, because the object file is
up-to-date. Consequently, the build system will fail to restore the deleted
interface file.

In Shake we can express this rule directly using operator \lst'&%>',
which defines a build rule with multiple outputs:

%(it actually turns out to be a
%consequence of the more powerful dependency model):

\begin{lstlisting}
["*.o", "*.hi"] &%> \[o, hi] -> do
    let hs = o -<.> "hs"
    need [hs]
    cmd "ghc" hc_opts hs
\end{lstlisting}

\subsection{Reducing concurrency\label{sec:ghc-pkg-db}}

As discussed in~\S\ref{sec:concurrency-reduction}, it is sometimes necessary to
reduce concurrency in a build system. As one example, GHC packages need to
be registered by invoking the \lst'ghc-pkg' utility. This utility mutates the
global state (package database) and hence at most one package can be registered
at a time, or the database is corrupted.

In Make, the solution is to introduce fake \emph{concurrency
reduction} dependencies. There are 25 GHC packages that require registration,
and in the old build system they all depended on each other in a chain, to
ensure that no simultaneous registrations occur. This solution works, but is
inelegant and inefficient.

There are really three problems. By over-sequentialising things we sacrifice
other parallelism that may be available in the system. We must over-constrain to
give a precise order in which the rules can be built. If one of the later ones
is available to build, we may need to stop it by waiting for another. Finally,
it's easy to miss such dependencies as they are not centralised. By accidentally
turning the chain into a tree we can end up with surprising failures.

In Shake, the solution is to use the \emph{resources} feature:

\begin{lstlisting}
packageDb <- newResource "package-db" 1
action $ withResource packageDb 1 $ cmd "ghc-pkg" ...
\end{lstlisting}

This snippet declares a global resource named \lst"packageDb" with quantity 1,
then calls \lst"withResource" asking for a single quantity before running the
\lst'ghc-pkg' utility. Provided all \lst'ghc-pkg' calls are suitably wrapped,
we will never run two instances simultaneously. Furthermore, thanks to the
availability of functions, we can abstract a function that executes
\lst'ghc-pkg' and applies the resource automatically, see
\S\ref{sec:abstractions} for how we do that.
% Note that build systems already schedule tasks to satisfy the implicit thread
% resource, so having user defined resources is an obvious generalisation.

\subsection{Dynamic dependencies\label{sec:dynamic-deps}}

Make, in common with many other build systems, works by
constructing a dependency graph and then executing it. This approach
has some benefits in that it is possible to analyse the graph ahead of
time, but it is limiting in some important ways.  It's common to run
into cases where parts of the dependency graph can only be known after
executing some other parts of the dependency graph.  Some examples of
this that occur in the GHC build system:

\begin{itemize}
\item GHC contains Haskell packages which have metadata specified in
  \lst'.cabal' files. We need to extract the metadata and generate
  \lst'package-data.mk' files to be included into the build system; this
  is done by a Haskell program called \lst'ghc-cabal', which is built by the
  build system. Hence we do not know how to build the packages until we have
  built and run the \lst'ghc-cabal' tool.

  One ``solution'' to this problem is to generate the \lst'.mk'
  files and check them into the repository whenever they change, but
  checking in generated files is ugly and introduces more failure
  cases when the generated files get out of sync.

\item The dependencies of a Haskell source file can be extracted by
  running the compiler, much like using \lst'gcc -M' on a C source
  file.  But the Haskell compiler is one of the things that we are
  building, so we cannot know the dependencies of many of the Haskell
  source files until we have built the stage 1 compiler.

\item Source files processed by the C preprocessor using
  \lst'#include' directives to include other files.  We can only
  know what the dependencies are by running the C preprocessor to
  gather the set of filenames that were included.
\end{itemize}

There are several other cases of such \emph{dynamic dependencies} in the build
system. Indeed, they arise naturally even when building a simple C program,
because the \lst'#include' files for a C source file are not known until the
source file is compiled or analysed by the C compiler, using something like
\lst'gcc -M'. Build systems that have a static dependency graph typically have
special-case support for this; for example, GNU Make will re-read
\lst'Makefiles' that change during building.  In the case of GNU Make,
this works for simple cases, but fails when there are dependencies
between the generated \lst'Makefiles', which arises in more complex cases.

The GHC build system works around this limitation of GNU Make by
dividing the build system into \emph{phases}, where each phase
generates the parts of the build system that are required by
subsequent phases. We managed to reduce the number of phases to
three, by carefully making use of GNU Make's automatic restarting
functionality where possible. However, explicit phases are a terrible
solution for several reasons:

\begin{itemize}
\item Phases reduce concurrency: we cannot start the next phase until
  the previous one is complete.
\item Knowing what targets to put in each phase requires a deep
  understanding of what the generated parts of the build system are
  and their dependencies, and this area of our build system is
  notoriously difficult to work on.  Diagnosing problems is
  particularly hard.
\item Even when doing an incremental build we have to perform all the
  phases, because explicit phases preclude dependency tracking across
  the phase boundary.  This means we have to work around slow
  incremental builds by having an explicit way for the user to say
  that they are sure nothing has changed that would require rebuilding
  the early phases (\lst'make fast').  Obviously this is less than
  ideal.
\end{itemize}

Shake supports dynamic dependencies natively, so these problems just
go away.  The \lst"need" function can introduce new dependencies
\emph{after observing the results of previous dependencies}\footnote{It has been
suggested that the dependency mechanism in Shake should rightly be called
\emph{Monadic dependencies} to contrast with the \emph{Applicative dependencies}
in Make. We agree. In an applicative computation the structure cannot depend on
the values flowing through container. In a monadic computation the structure can
depend on the values.}. For example, when building a Haskell package, we first
\lst"need" the \lst'ghc-cabal' tool, then we run it on the package's \lst'.cabal' file to
extract the package metadata, and then we consult the result to determine what
needs to be done to build the package.

The existence of \lst"need" means that Shake cannot construct the
dependency graph ahead of time, because it doesn't know the full
dependency graph until the build has completed, but this is not a
limitation in practice (while the lack of dynamic dependencies really
\emph{is} a limitation requiring painful workarounds).

% SimonM: I wasn't really sure what this paragraph was trying to say
% While the existence of the \lst"need" fundamentally alters the expressive
% power of Shake, for users it is merely a relaxation of an unnecessary rule
% (that \lst"need" must be the in a rule), so just makes things easier. At the
% build system level it requires many changes, the lack of a static dependency
% graph, termination checking as it executes, the ability to suspend partially
% executed rules etc. Fortunately, as users we do not need to care.

Note, Redo (\S\ref{sec:redo}) is the only other clearly monadic build
system, while it can be encoded in SCons using the dependency rules.

% \subsection{Fine-grain dependencies\label{sec:fine-grain-deps}}
% 
% There are many instances where an action depends on a part of a file. As a
% compelling example in GHC, there are many configuration settings provided in a
% file \lst"system.config", which reads:
% 
% \begin{lstlisting}
% system-ghc = /usr/bin/ghc
% system-gcc = /usr/bin/gcc
% ...
% \end{lstlisting}
% 
% \vspace{-2mm}
% When executing a command we typically depend on only a handful of lines in this
% large configuration file. However, using simple file dependencies, we have to
% depend on the whole file. As a result, we end up rebuilding more than is
% necessary (see~\S\ref{sec:ghc}).
% 
% One solution is to create a single file for each setting, but in GHC the whole
% \lst"system.config" file is generated by the \lst'configure' script, which makes
% this challenging. Since a single file is being generated, the next best solution
% is to chop it up into multiple small files, to provide accurate fine-grained
% dependencies.
% 
% This approach does not work in Make. We can produce individual
% files, but the individual files must all depend on \lst"system.config", so they
% update appropriately. In the rule to generate the fragments, if we always generate the
% new file, then we do not break the dependencies -- everything still has an
% ultimate dependency on the whole file. If in a rule we chose to avoid generating
% the small file then the file always looks dirty (since its timestamp is older
% than that of \lst"system.config", which Make considers to signify dirty).
% 
% Fortunately, Shake supports polymorphic dependencies that solve this problem,
% as discussed in~\S\ref{sec:polymorphic}. This brings dramatic improvements -- on
% real build systems we regularly see build times of a couple of seconds, which
% would have been minutes without this feature (e.g., see~\S\ref{sec:ghc}). This
% feature is also available in \lst'Tup' and \lst'Ninja', although in both cases
% requires explicitly enabling per rule.

% (using \lst'stat' in \lst'Ninja', and \texttt{\^} in Tup).

% Using the same approach in Shake we can simply avoid writing out the small file
% if it has not changed. Shake considers a file to be rebuilt if the rule runs
% successfully, but only considers the file \emph{changed} if its value changed.
% As a result, Shake can start down a build tree and then later avoid that build
% tree. This feature is built into the core of Shake, and means that, unlike
% Make, we can split a single dependency into fine-grained dependencies
% successfully.

% \subsection{Everything is a file}
%
% Once fine grained dependencies are available, they can quickly become
% pervasive. Using a build system based on filenames as keys and files as values
% we effectively reach the global store problem we criticised in \S? (particularly
% if we consider dyanmic dependencies from \S? to be \lst"$" dereferencing!).
% While not critical, Shake supports an Oracle feature, to allow us to make the
% keys and values arbitrary Haskell types, and store them in the main Shake
% database. Thanks to such a technique we can have an oracle per rule we create to
% track changing information, without necessarily doubling the number of files
% present in the build system. As an example:
%
% \begin{lstlisting}
% newtype ConfigKey = ConfigKey String
%     deriving (Show,Typeable,Eq,Hashable,Binary,NFData)
% rules = do
%     addOracle $ \(ConfigKey x) -> do
%         src <- readFile' "system.config"
%         Map.lookup (parseFile src) x
%
%     ... askOracle (ConfigKey x) ...
% \end{lstlisting}
%
% Here we define our own rule of type \lst"ConfigKey" which reads from the
% configuration and stores just that one field. Since this also participates in
% unchanging rules we automatically get rebuild avoidance if the value does not
% change.

\subsection{Real code\label{sec:real_code}}

Much of a build system is calling out to programs that execute real code, e.g.
compilers. But there is some stuff that requires custom code, for example
splitting command line arguments that exceed a certain size. In the old
build system we used \lst'xargs', which alas does not work consistently between
different OS versions, and does both more than we want and less than we want.
Fortunately, with Haskell at our disposal, we can write:

\begin{lstlisting}
-- | @chunksOfSize size strings@ splits a given list of strings
-- into chunks not exceeding @size@ characters. If that is
-- impossible, it uses singleton chunks.
chunksOfSize :: Int -> [String] -> [[String]]
chunksOfSize n = repeatedly $ \xs ->
    let#\,#i = length $ takeWhile#\,#(<=#\,#n) $ scanl1#\,#(+) $ map#\,#length#\,#xs
    in splitAt (max 1 i) xs
\end{lstlisting}

\noindent Writing a small function is easy in Haskell, and even easier to test
(we test this function using QuickCheck). Writing it in \lst'bash' would be infeasible.
Reducing the specific behaviours required from external tools leads to
significantly less cross-platform concerns.

\subsection{Summary}

The unnecessary complexities described in this Section have a big consequence in
terms of complexity and performance. They obscure the underlying real stuff in
the build system. Simply removing these already gets us to a much better state.

The main lessons we have learnt are:

\begin{itemize}
\item Abstraction is a powerful and necessary tool. Lack of good abstraction
mechanisms is a significant cause of the complexity of previous attempts in
Make. Functional programming provides good abstractions which in turn allow
reducing the complexity at each level, while still obtaining a more powerful
result. Marrying build systems and functional programming works well.
\item Use a build system that can express the necessary dependencies. We make
use of multiple outputs~\S\ref{sec:multiple-outputs},
resources~\S\ref{sec:ghc-pkg-db}, dynamic dependencies~\S\ref{sec:dynamic-deps},
fine-grain dependencies~\S\ref{sec:fine-grain-deps}. While some of these
features are only used in a few places (e.g., resources), their absence would
require pervasive workarounds, and a significant increase in the overall complexity.
\end{itemize}

% \subsection{Abstracting over patterns of dependencies}
%
% \todo{This was moved from Section 5.}
%
% By embedding the build system in Haskell we get access to its rich abstraction
% facilities. Examples:
% \begin{itemize}
%   \item Use general predicates instead of file patterns. For example, several
%   files are built by calling \texttt{genprimopcode}, which can be captured by a
%   single build OR-rule:
%
% \begin{lstlisting}
% [ "autogen/GHC/Prim.hs"
% , "GHC/PrimopWrappers.hs"
% , "*.hs-incl" ] |%> \file -> do ...
% \end{lstlisting}
%
%   \item Build rules with multiple outputs (AND-rules)
%   \item Compute build arguments from target filename
% \end{itemize}

\section{Abstractions\label{sec:abstractions}}

% \begin{itemize}
%   \item Example shared with Background to Shake section
%   \item top-level, starting with main = buildPackage etc.
% \end{itemize}

In the previous section we covered how Shake helps us sidestep the unnecessary
complexities inherent in a large-scale build system. In this section we focus on
the remaining complexities on the way to a successful build system. Before
starting, it is useful to explain our goals. We want the resulting system to be
maintainable, in addition to being fast and correct. The following
aspects are particularly important for maintainability of a large-scale build
system:

\begin{itemize}
\item The code should be simple and direct, talking about concepts familiar to
developers, such as files that are built, tools that are used to build
them, and their configuration settings. These concepts need first-class
representation, not merely strings.
\item The code should permit debugging. There are many obscure configuration
options inherent in any large software project, and the build system is where
such knowledge usually lives. As an example, to include profiling information
when compiling a Haskell source file with GHC we need to pass the \lst'-prof' flag.
We want users to be able to easily determine what flags were used, as well
as why and where specifically that decision was made.
\item The code should permit configuration. Many GHC users work with GHC in
different modes and in different environments. Common configuration settings
include turning on/off documentation, choosing different sets of optimisation flags,
etc.
\end{itemize}

The solution was not obvious to us at first, but we ended up building a DSL for
expressions, and an interpreter that evaluates it to run the build.
% Where the DSL would be unnecessarily complex we opt for a direct
% implementation.
The configuration language is tracked (if it changes the affected build rules
are rerun), has provenance (using the new \lst'?location :: Location' feature in
GHC) and permits easy configuration, for example:

\begin{lstlisting}
builder Ghc ? way profiling ? arg "-prof"
\end{lstlisting}

\noindent The expression adds the \lst'-prof' argument to the command line when
building a file with GHC in the \lst'profiling' way.

\newcommand{\itab}[1]{\hspace{0em}\rlap{#1}}
\newcommand{\tab}[1]{\hspace{.1\textwidth}\rlap{#1}}
\newcommand{\ctab}[1]{\hspace{.031\textwidth}\rlap{#1}}
\newcommand{\ptab}[1]{\hspace{.074\textwidth}\rlap{#1}}
\newcommand{\cotab}[1]{\hspace{.064\textwidth}\rlap{#1}}
\newcommand{\ttab}[1]{\hspace{.058\textwidth}\rlap{#1}}
\newcommand{\tytab}[1]{\hspace{.052\textwidth}\rlap{#1}}
\newcommand{\atab}[1]{\hspace{.102\textwidth}\rlap{#1}}

\begin{figure}
\begin{lstlisting}
#\vspace{-7mm}#
#\line(1,0){240}#
data PackageType = Library | Program#\hspace{15mm}\textbf{\emph{Build types}}#

data Package = Package
#\itab{~~~~\{} \ctab{pkgName} \ptab{:: PackageName}#
#\itab{~~~~,} \ctab{pkgPath} \ptab{:: FilePath}#
#\itab{~~~~,} \ctab{pkgType} \ptab{:: PackageType \}}#

data Stage = Stage0 | Stage1 | Stage2 | Stage3 deriving#\,#Enum

newtype Way = ... deriving Eq

data Context = Context
#\itab{~~~~\{} \ctab{stage} \cotab{:: Stage}#
#\itab{~~~~,} \ctab{package} \cotab{:: Package}#
#\itab{~~~~,} \ctab{way} \cotab{:: Way \}}#

data Builder = Alex
                  | Ar
                  | GenPrimopCode
                  | Ghc Stage
                  | Haddock
                  ... (22 more builders)

data Target = Target
#\itab{~~~~\{} \ctab{context} \ttab{:: Context}#
#\itab{~~~~,} \ctab{builder} \ttab{:: Builder}#
#\itab{~~~~,} \ctab{inputs} \ttab{:: [FilePath]}#
#\itab{~~~~,} \ctab{outputs} \ttab{:: [FilePath] \}}#
#\vspace{-5mm}#
#\line(1,0){240}#
type Expr a = ReaderT Target Action a#\hspace{14mm}\textbf{\emph{Expressions}}#

newtype Diff a = Diff { fromDiff :: a -> a }

type Args = Expr (Diff [String])

append :: [String] -> Args
append as = return $ Diff (<> as)

remove :: [String] -> Args
remove as = return . Diff $ filter (`notElem` as)

arg :: String -> Args
arg = append . return

interpret :: Target -> Args -> Action [String]
interpret target args = do
    diff <- runReaderT args target
    return $ fromDiff diff mempty
#\vspace{-5mm}#
#\line(1,0){240}#
type Predicate = Expr Bool#\hspace{31mm}\textbf{\emph{Predicates}}#

package :: Package -> Predicate
package p = do
    c <- asks context
    return $ p == Context.package c

#\itab{stage} \tytab{:: Stage} \atab{$\rightarrow$ Predicate}#
#\itab{way} \tytab{:: Way} \atab{$\rightarrow$ Predicate}#
#\itab{builder} \tytab{:: Builder} \atab{$\rightarrow$ Predicate}#
#\itab{file} \tytab{:: FilePattern} \atab{$\rightarrow$ Predicate}#

(?) :: Monoid a => Predicate -> Expr a -> Expr a
predicate ? expr = do
    bool <- predicate
    if bool then expr else return mempty
#\vspace{-4mm}#
\end{lstlisting}
\caption{GHC build system abstractions\label{fig:abstractions}}
\end{figure}

\subsection{Programming model}

GHC source code is split into logical units, or \emph{packages}. We model
packages with the \lst'Package' type, see Figure~\ref{fig:abstractions}
(the \emph{Build types} subfigure). A package is identified by a unique
\lst'PackageName' and a \lst'FilePath' pointing to its location in the source
tree. A GHC package can be a library (e.g., \lst'array') or a program (e.g.,
\lst'genprimopcode'), which is captured by \lst'PackageType'. There are 32
libraries and 18 programs (the latter includes the GHC itself and various
utilities).

A package can be built multiple \emph{ways}, for example, to produce a library
with or without profiling information. This is captured by an opaque
type \lst'Way' inhabited by values such as \lst'vanilla' (the simplest
possible way), \lst'profiling' (with profiling information), \lst'debug'
(with debug information), and many others (there are 18 ways in total). Some
ways can be combined, e.g., \lst'debugProfiling'; however, not all combinations
are allowed or currently supported. By making \lst'Way' opaque we make it easier
to add new ways or change their internal representation, something that would be
impossible to achieve in Make, where no information hiding is possible. 

In addition to different build ways, each package can be built by several
versions of GHC, which leads to the notion of \emph{stages}.

In \lst'Stage0' we use the \emph{bootstrap} GHC, i.e. the one that is
installed in the system. During this stage we build \lst'Stage1' GHC, an
intermediate compiler that still lacks many features. It is used during the
following \lst'Stage1' for building a fully-featured \lst'Stage2' GHC, the
primary goal of the build system. We sometimes also build \lst'Stage3' GHC as
a self-test: the object code of \lst'Stage2' and \lst'Stage3' compilers
should be the same.

\lst'Stage', \lst'Package' and \lst'Way' form a GHC-specific
\emph{build context} represented by type \lst'Context'. A typical GHC build
rule, such as \lst'compilePackage', depends on the context as follows: it
uses an appropriate compiler version (e.g., the bootstrap compiler in
\lst'Stage0'), produces object files with different extensions (e.g., vanilla
\lst'*.o' or profiled \lst'*.p_o' object files), puts build artefacts
into an appropriate directory (e.g., \lst'stage1/libraries/base'), etc.
There are $4 \times 40 \times 18 = 2880$ different possible contexts.

A typical build system invokes several build tools, or \emph{builders}, such as
compilers, linkers, archivers, etc., some of which may also be built by the
build system itself. The builders are captured by the \lst'Builder' type. It
is useful to distinguish \emph{internal} and \emph{external} builders, i.e.
those that are build by the build system and those which are installed in the
system, respectively. Function \lst'builderProvenance' returns the stage
during which an internal builder is built, the way it is built, and the package
containing the sources (all captured by a \lst'Context'); \lst'Nothing' is
known about the provenance of external builders.

\begin{lstlisting}
builderProvenance :: Builder -> Maybe Context
builderProvenance = \case
    GenPrimopCode -> Just $
        Context Stage0 genprimopcode vanilla
    Ghc stage ->
        if stage == Stage0 then Nothing
        else Just $ Context (pred stage) ghc vanilla
    Haddock -> Just $
        Context $ Stage2 haddock vanilla
    ...
    _ -> Nothing
\end{lstlisting}

In particular, one can see that \lst'Ghc Stage0' is an external builder,
\lst'Ghc Stage1' is an internal one built from package \lst'ghc'
during \lst'Stage0', \lst'Haddock' is built in \lst'Stage2', etc. There are
27 builders and 16 of them are internal. Furthermore, some builders are
\emph{optional}, e.g., \lst'HsColour', which (if installed) is used to
colourise Haskell code when building documentation.

Each invocation of a builder is fully described by a \lst'Target', which
comprises a build \lst'Context', a \lst'Builder', a list of input and
a list of output files. 3748 targets are built when building \lst'Stage2' GHC
with documentation (with vanilla and profiled libraries). Consider the following
invocation of builder \lst'Alex' as an example:

\begin{lstlisting}
alex -g --latin1 compiler/parser/Lexer.x
                -o compiler/stage1/build/Lexer.hs
\end{lstlisting}

\noindent The corresponding \lst'Target' is:

\begin{lstlisting}
lexerTarget = Target
#\itab{~~~~\{} \ctab{context} \ttab{= Context Stage1 compiler vanilla}#
#\itab{~~~~,} \ctab{builder} \ttab{= Alex}#
#\itab{~~~~,} \ctab{inputs} \ttab{= ["compiler/parser/Lexer.x"]}#
#\itab{~~~~,} \ctab{outputs} \ttab{= ["compiler/stage1/build/Lexer.hs"] \}}#
\end{lstlisting}

\noindent By examining \lst'lexerTarget' it is possible to compute the full
command line for building \lst'compiler/stage1/build/Lexer.hs':
\begin{itemize}
  \item The builder is \lst'Alex'. We lookup the right command
  \lst'alex' in the configuration file \lst'system.config', and use the
  following command line expression: \lst'-g <input> -o <output>'.
  \item The package is \lst'compiler'. We know that we need to add
  flag \lst'--latin1' to the above expression.
  \item We know how to substitute \lst'<input>' and \lst'<output>' in the
  above expression.
  \item We ignore \lst'stage' and \lst'way' because they are not relevant
  in this particular case. However, we record them in the build log.
\end{itemize}

\noindent A build system typically contains many such computations (at least one
for each builder) and it is important to provide a terse and readable notation to
describe them. After experimenting with several abstractions, we converged on
\emph{expressions} defined in the next subsection.

\subsection{Expressions}

An expression \lst'Expr a' is a computation that produces a value of type
\lst'Action a' and can read the current build \lst'Target', as shown in
Figure~\ref{fig:abstractions} (see the \emph{Expressions} subfigure). For example,
the following expression computes command line arguments for invoking \lst'Alex':

\newcommand{\altab}[1]{\hspace{.05\textwidth}\rlap{#1}}

\begin{lstlisting}
alexArgs :: Expr [String]
alexArgs = do
#\itab{~~~~pkg} \altab{$\leftarrow$ asks Target.package}#
#\itab{~~~~src} \altab{$\leftarrow$ asks Target.inputs}#
#\itab{~~~~out} \altab{$\leftarrow$ asks Target.outputs}#
    return $ [ "-g" ]
          #\!#++ [ "--latin1" | pkg == compiler ]
          #\!#++ [ head src ]
          #\!#++ [ "-o", head out ]
\end{lstlisting}

There is a room for improvement of the \lst'alexArgs' expression. For
example, conditional arguments like \lst'--latin1' are very common in build
systems and a better way to express them is clearly desirable. We use Boolean
\emph{predicates} of type \lst'Expr Bool' to achieve this, see
Figure~\ref{fig:abstractions} (the \emph{Predicates} subfigure). In particular, we
use \lst'package :: Package -> Predicate' to check whether a particular package
is currently being built. For example, predicate \lst'package compiler'
returns \lst'True' when the current target belongs to the \lst'compiler'
package.

Operator \lst'(?) :: Monoid a => Predicate -> Expr a -> Expr a' is a
convenient shortcut for applying a predicate to an expression that computes a monoidal value, such
as \lst'[String]'. For example, the following expression returns
\lst'--latin1' when the current target belongs to the compiler package and an
empty expression otherwise:

\begin{lstlisting}
latin1 :: Expr [String]
latin1 = package compiler ? return ["--latin1"]
\end{lstlisting}

\noindent Note, expressions computing monoids themselves form a monoid:

\begin{lstlisting}
instance Monoid a => Monoid (Expr a) where
    mempty #\hspace{1.5mm}#= return mempty
    mappend = liftM2 mappend
\end{lstlisting}

Predicates and monoidal expressions are a powerful combination with many useful
laws that allow to reason about them:
\begin{enumerate}
  \item \itab{Absorption:} \tab{\lst'p ? mempty === mempty'}
  \item \itab{Distributivity:} \tab{\lst'p ? (e <> f) === p ? e <> p ? f'}
  \item \itab{Conjunction:} \tab{\lst'p ? q ? e === (p &&& q) ? e === q ? p ? e'}
  \item \itab{Disjunction:} \tab{if \lst'p &&& q === False' then} \vspace{1mm}\\
  \lst'    p ? e <> q ? e === (p ||| q) ? e === q ? e <> p ? e'
  \item \itab{Complement:} \tab{\lst'p ? e <> !!!p ? e === e'}
\end{enumerate}

All expressions need to be modifiable by users of the build system. We therefore
need to provide a way not only to add new arguments to an expression, but also
to modify and remove them. A simple way to achieve this is to switch to difference
list expressions, represented by type \lst'Expr (Diff a)', which is used to
construct values of type \lst'Diff a' with the following monoid instance:

\begin{lstlisting}
instance Monoid (Diff a) where
    mempty #\hspace{24mm}#= Diff id
    mappend (Diff x) (Diff y) = Diff $ y . x
\end{lstlisting}

\noindent The reverse order of function composition \lst'y . x' ensures that
when two \lst'Expr (Diff a)' computations are combined \lst'c1 <> c2', then
\lst'c1' is applied first and \lst'c2' is applied second.

The following functions can be used to append and remove items to/from a
difference list:

\begin{lstlisting}
append :: Monoid a => a -> Expr (Diff a)
append x = return $ Diff (<> x)

remove :: Eq a => [a] -> Expr (Diff [a])
remove xs = return . Diff $ filter (`notElem` xs)
\end{lstlisting}

\newcommand{\tabx}[1]{\hspace{.106\textwidth}\rlap{#1}}
\newcommand{\taby}[1]{\hspace{.103\textwidth}\rlap{#1}}
\newcommand{\tabz}[1]{\hspace{.24\textwidth}\rlap{#1}}

\begin{figure}
\begin{lstlisting}
#\vspace{-7mm}#
#\line(1,0){240}#
#\itab{data Stage} \tabx{= Stage0} \taby{| Stage1}\hspace{29mm}\textbf{\emph{Build~types}}#
#\itab{data Package} \tabx{= Array} \taby{| Base}#
#\itab{data Way} \tabx{= Vanilla} \taby{| Profiling}#
#\itab{data Builder} \tabx{= Ghc Stage} \taby{| Ar}#
#\vspace{-5mm}#
#\line(1,0){240}#
ghcArgs, arArgs, userArgs, allArgs :: Args#\hspace{7.8mm}\textbf{\emph{Command~line}}#
ghcArgs = builder Ghc ? mconcat#\hspace{23.3mm}\textbf{\emph{arguments}}#
    [ arg "-O2"
    , way Profiling ? arg "-prof"
    , file "//base/GHC/IO.*" ? arg "-funbox-strict-fields" 
    , arg "-c", arg =<< getInput 
    , arg "-o", arg =<< getOutput ]

arArgs = builder Ar ? mconcat
    [ arg "q"
    , arg =<< getOutput
    , append =<< getInputs ]

userArgs = package Array ? arg "-Wall"

allArgs = ghcArgs <> arArgs <> userArgs
#\vspace{-5mm}#
#\line(1,0){240}#
build :: Target -> Action ()#\hspace{32mm}\textbf{\emph{Build rules}}#
build target@Target {..} = do
    path <- builderPath builder
    args #\,#<- interpret target allArgs
    need [path]
    cmd [path] args

path :: Context -> FilePath
path context@Context#\,#{..} = show#\,#stage </> pkgName#\,#package

#\itab{osuf :: Way $\rightarrow$ String} \tabz{asuf :: Way $\rightarrow$ String}#
#\itab{osuf Vanilla~~~= ".o"} \tabz{asuf Vanilla~~~= ".a"}#
#\itab{osuf Profiling = ".p\_o"} \tabz{asuf Profiling = "\_p.a"}#

lookupDependencies :: Context#\,#->#\,#FilePath#\,#->#\,#Action#\,#[FilePath]

lookupSources :: Context -> Action [FilePath]

compilePackage :: Context -> Rules ()
compilePackage context@Context {..} = 
    path context </> "*" ++ osuf way %> \obj -> do
        let src = obj -<.> "hs"
        deps <- lookupDependencies context obj
        need $ src : deps
        build $ Target context (Ghc stage) [src] [obj]

buildPackageLibrary :: Context -> Rules ()
buildPackageLibrary context@Context {..} = 
    path context </> "*" ++ asuf way %> \a -> do
        srcs <- lookupSources context
        let objs = [ src -<.> osuf way | src <- srcs ]
        need objs
        build $ Target context Ar objs [a]
#\vspace{-5mm}#
#\line(1,0){240}#
main :: IO ()#\hspace{59.5mm}\textbf{\emph{Main}}#
main = shake shakeOptions $ do
    let contexts = Context <$> 
        [Stage0,#\,#Stage1]#\,#<*>#\,#[Array,#\,#Base]#\,#<*>#\,#[Vanilla,#\,#Profiling]
  
    want [#\,#path#\,#c </> "HSlib" ++ asuf#\,#(way#\,#c) | c <- contexts#\,#]

    traverse_ compilePackage      #\hspace{0.2mm}#contexts
    traverse_ buildPackageLibrary contexts
#\vspace{-4mm}#
\end{lstlisting}

\caption{Example of a build system\label{fig:example-abstractions}}
\end{figure}
We are now ready to introduce \lst'Args', a type of expressions for
constructing command line arguments in the build system. In addition to the
above generic functions (whose specialised versions are shown in
Figure~\ref{fig:abstractions}), it is equipped with function
\lst'arg :: String -> Args' for injecting simple \lst'String' arguments into
\lst'Args'. With these abstractions in place, we can construct command line
arguments for \lst'Alex' as follows:

\begin{lstlisting}
alexArgs :: Args
alexArgs = mconcat [ arg "-g"
#\hspace{28.8mm}#, package compiler ? arg "--latin1"
#\hspace{28.8mm}#, arg =<< getInput
#\hspace{28.8mm}#, arg "-o", arg =<< getOutput ]
\end{lstlisting}

\noindent Here \lst'getInput :: Expr FilePath' and
\lst'getOutput :: Expr FilePath' are expressions that check that
\lst'Target.inputs' and \lst'Target.outputs' contain exactly one element and
return it.

The resulting \lst'alexArgs' expression is terse and readable. All
distracting plumbing details have been abstracted away so that the designers and
users of the build system could focus on what matters. \todo{Also mention
Packages and Ways expressions.}

We compose all command line arguments into \lst'allArgs :: Args' expression,
applying custom user modifications \lst'userArgs' in the very end, hence
allowing the user to override any default setting:

\begin{lstlisting}
allArgs :: Args
allArgs = mconcat [ builder Alex ? alexArgs, ..., userArgs ]
\end{lstlisting}

The resulting expression is used in the \lst'build' function that is responsible
for building a given \lst'Target':

\begin{lstlisting}
build :: Target -> Action ()
build target@Target {..} = do
    path <- builderPath builder
    args #\,#<- interpret target allArgs
    need [path]
    checkArgsHash target
    cmd [path] args
\end{lstlisting}

\noindent We clarify each statement of the \lst'build' function below:
\begin{itemize}
  \item Firstly, \lst'builderPath :: Builder -> Action FilePath' determines the
  path to the builder depending on its provenance and the contents of the
  \lst'system.config' file.
  \item We \lst'interpret' the \lst'allArgs' expression w.r.t. to the
  \lst'target', and obtain the list of \lst'String' arguments to be passed to
  the builder. See Figure~\ref{fig:abstractions} for the implementation of
  \lst'interpret'.
  \item We \lst'need' the builder to make sure it is up-to-date. Some builders
  are built by the build system, e.g. \lst'genprimopcode',
  \lst'ghc-cabal', stage1 GHC, so it is important to rebuild them (as well as
  all dependent targets) if need be.
  \item Function \lst'checkArgsHash :: Target -> Action ()' checks whether the
  \lst'target' needs to be rebuilt because the command line computed from
  \lst'allArgs' expression has changed since the previous build. If it changed,
  the target is rebuilt even if all its dependencies are up-to-date. This tracks
  changes both in the environment and in the build system itself.
  \todo{\textbf{Andrey}: Explain how checkArgsHash works?}
  \item Finally, we invoke the builder with appropriate arguments using Shake's
  \lst'cmd' function.
\end{itemize}

\subsection{A simple build system example}

Figure~\ref{fig:example-abstractions} shows a simple build system that uses the
abstractions introduced in this section.

\todo{\textbf{Andrey}: Add description of the example (0.5 columns).}

\section{Shaking up GHC\label{sec:ghc}}

In this section we report on our experience on applying the techniques presented
so far to building a large-scale software project: the Glasgow Haskell Compiler.

\todo{Andrey: list current limitations.}

\subsection{Analysis of common use-cases}
\begin{table*}[t]
\centering
\begin{tabular}{c | p{56mm} || p{50mm} | p{50mm}}
~
& \textbf{Use case}
& \textbf{Old build system} based on \make{}
& \textbf{New build system} based on Shake
\\
\hline
\hline
1 & Full clean build
& Everything is built, N sec \hfill \checkmark
& Everything is built, M sec \hfill \checkmark
\\
\hline
2 & Zero build \hspace{6.4mm}
& Nothing is rebuilt, 12.3 sec \hfill \checkmark
& Nothing is rebuilt, 3.7 sec \hfill \checkmark
\\
\hline
3 & Add comment: \textsf{libraries/base/Prelude.hs}
& \textsf{Prelude.o}, \textsf{base} library, and all
\newline dependent binaries are rebuilt
& Only \textsf{Prelude.o} is rebuilt \hfill \checkmark
\\
\hline
4 & Modify code: \hspace{1.75mm}\textsf{libraries/base/Prelude.hs}
& \textsf{Prelude.o} and all its dependencies \hfill \checkmark \newline
are rebuilt (almost everything)
& \textsf{Prelude.o} and all its dependencies \hfill \checkmark \newline
are rebuilt (almost everything)
\\
\hline
5 & Add comment: \textsf{utils/ghc-cabal/Main.hs}
& Almost everything is rebuilt, since \newline \textsf{package-data.mk} files
are changed & All \textsf{ghc-cabal} rules are rerun (due \newline
to non-deterministic GHC builds)
\\
\hline
6 & Modify code: \hspace{1.75mm}\textsf{utils/ghc-cabal/Main.hs}
& Almost everything is rebuilt
& Only the targets affected by the \newline change are rebuilt \hfill
\checkmark
\\
\hline
7 & Modify the build system: pass \textsf{-O2} when compiling Stage2 GHC
& Nothing is rebuilt, as build system \newline is not tracked
& Stage2 GHC and all its \newline dependencies are rebuilt \hfill \checkmark\\
\hline
8 & Modify the build system without changing command line arguments of build
tools
& Nothing is rebuilt
& Nothing is rebuilt
\\
\hline
9 & Switch to a \textsf{git} branch and back 
& Some files are rebuilt
& Nothing is rebuilt \hfill \checkmark
\\
\hline
10 & Change path to \textsf{gcc}
& Everything is rebuilt
& Files depending on \textsf{gcc} are rebuilt \hfill \checkmark
\\
\end{tabular}
\caption{Comparison of GHC build systems on common use cases
(using \textsf{quick} build flavour). Checkmarks \checkmark indicate expected
behaviour.}
\label{tab:use-cases}
\end{table*}


In this section we discuss several use-cases of the GHC build system, which
are fairly typical for build systems in general.

\subsubsection{Change nothing and rebuild}

Running a build system twice in a row should be equivalent to only running it
once. Moreover the second build should do nothing, as quickly as possible.
Such a do-nothing build is referred to as a \emph{zero build}. The time it takes
represents internal overheads of the build system: scanning the file system,
reading the database, etc.

\textbf{Outcome:} Both the old and the new build systems pass this test
successfully. Notably, the new one does this 3 times faster, as reported in
\S\ref{sec:benchmarks}.

\subsubsection{Change a source file and rebuild}

\begin{lstlisting}
$ build
$ build
\end{lstlisting}

Both \make{} and Shake work. Shake does fewer rebuilds. Several examples with
different rebuild chains.

\subsubsection{Add comments to \texttt{ghc-cabal} and rebuild}

Both \make{} and Shake work. \make{} ends up rebuilding everything. Shake only
reruns \texttt{ghc-cabal} and stops, as generated files are not changed.

\subsubsection{Change a command line in the build system and rebuild}

\make{} doesn't track such changes. Shake reruns only necessary rules.
Also show how changing a command line looks like in \make{} and in Shake.

\subsubsection{Switch between \texttt{git} branches and rebuild}

Consider the following sequence of commands, which can occur when working on
multiple \texttt{git} branches simultaneously:

\begin{lstlisting}
$ build
$ git checkout -b test
$ touch libraries/base/Prelude.hs
$ git checkout master
$ build
\end{lstlisting}

We expect the second \texttt{build} to do nothing, since the \texttt{master}
branch hasn't changed. However, \texttt{git branch master} changes modification
time of all files that were updated between branches, which forces \make{} to
rerun all dependent rules. Shake's ability to track file contents in addition to
modification time allows us to avoid such unnecessary rebuils, see
\S\ref{sec:file-contents}.

\textbf{Outcome:} the old build system does a partial rebuild, the new build
system rebuilds nothing.

\subsubsection{Change path to \texttt{gcc} and rebuild}

Both \make{} and Shake work. \make{} ends up rebuilding everything. Shake reruns
only necessary rules.

\subsubsection{Change build rules and rebuild}

Both \make{} and Shake don't notice this and fail to rebuild. Checking
equivalence is hard. Future work.

\subsection{Benchmarks\label{sec:benchmarks}}

\todo{Neil: your profiling results can go here}

Quantitative results: benchmarking, profiling, achieved parallelism.

We used parallelism to speed up some phases. Show profiles showing ghc-cabal was
a bottleneck.

\subsection{Things lint spotted}

\todo{Neil: I think I have only one good example: result of getDirectoryFiles
changed. Do we want to discuss it here? Maybe somewhere in the Shake section
instead?}

Most useful one is if two people whack the same output. Other thing is where
doing all .hs files in a dir, and then you generate one. Should this go
somewhere else? Perhaps in S5.

\subsection{Summary}

\todo{How we used better tools and abstractions to do better.}

\todo{Quantitative results: are we faster? Fewer lines of code? Fewer system
dependencies?}

\todo{Qualitative results: easier to understand and maintain. Can we
say this? Can we say that the new build system works in more environments,
because we have fewer dependencies?}


\subsection{Performance notes}

Zero build Shake

\begin{lstlisting}
shakeArgsWith                        0.001s    0%
Function shake                       0.156s    4%  =
Database read                        0.329s   10%  ===
With database                        0.021s    0%
Running rules                        2.502s   77%  =========================
Pool finished (7639 threads, 8 max)  0.001s    0%
Lint checking                        0.202s    6%  ==
Total                                3.211s  100%
Build completed in 0:04m

real    0m3.700s
\end{lstlisting}

Zero build with Make, real    0m12.342s


\section{Related Work\label{section-review}}

This paper is about writing build systems at scale, a subject without much
literature since \citet{miller:recursive_make}. When
\citet{mcintosh:build_maintenance_effort} studied software maintenance they
found that build systems can take up to 27\% of
development effort, and that improvements to the build system rapidly paid off.
Recently \citet{martin:make_it_simple} surveyed which Make features are used,
and then \citet{martin:maintenance_complexity_makefiles} classified them by
complexity -- unsurprisingly they found that as Makefiles grow, their complexity
increases, and that the features required for hand written Makefiles are those
which are most complex. In the remainder of this section we focus on some of the
features found in other build systems which could be useful at scale.

\subsection{Embedded language}

A build system can either be specified using structured metadata, e.g. Bazel
\cite{bazel}, or embedded into a standard programming language -- for example
SCons in Python \cite{scons}, Pluto in Java \cite{pluto} and Jenga in OCaml
\cite{jenga}. For complex bespoke build systems, embedding into a language
allows both complex operations (\S\ref{sec:real_code}) and better abstractions
(\S\ref{sec:abstractions}) -- essentially allowing us to write most of our build
system in a domain language tailored to our specific project.

Even sticking to Haskell as the embedded language, there are a surprisingly
large number of libraries implementing a dependency aware build system -- we
know of eleven in addition to Shake (Abba, Blueprint, Buildsome, Coadjute, Cake
$\times$ 2, Hake, Hmk, Nemesis, OpenShake and Zoom). Of these, the two Cake
libraries and OpenShake are based on an early presentation of the principles behind Shake.

\subsection{Advanced dependencies}

We have found that while powerful dependencies might only be used in a few
places, if they are missing the workarounds can be pervasive
(\S\ref{sec:dynamic-deps}). A few build systems support resources, e.g.
Ninja \cite{ninja}, and several support monadic dependencies (e.g. Redo
\cite{redo}, Jenga, Pluto, SCons). A few build systems directly support
dependency features more powerful than Shake, for example Pluto supports rules
that run until a fixed-point is reached and rules whose output is not known in
advance. These features can be encoded in Shake, but are not present natively.

\subsection{Automatic dependency management}

In both Shake and Make, all dependencies must be declared explicitly. However,
in build systems such as Tup \cite{tup} and Buildsome \cite{buildsome}, some
dependencies are automatically captured by monitoring program execution, albeit
only \emph{after} the dependency has been used (like \lst'needed' in
\S\ref{sec:needed}). The Fabricate tool \cite{fabricate} takes a unique approach
to defining build systems, providing a series of steps that run sequentially,
but are skipped if their automatically-detected inputs have not changed.
Unfortunately no cross-platform APIs are available to detect used dependencies,
so such tools are all limited in which platforms they support.

\subsection{Build clusters}

The build systems Bazel and Buck \cite{buck} are used at Google and Facebook
respectively, both operating at sizes significantly beyond that of the GHC build
system (reportedly billions of lines of code). Both systems take a metadata
approach, with various rule types baked in. As an example, the \lst'cxx_binary'
rule builds a C/C++ binary given a list of source files and dependencies,
taking care of suitable build flags and conventions, much like
\lst'buildPackage' from~\S\ref{sec:build-example} but a lot more feature-rich.
The disadvantage of such an approach is that the available rules are fixed,
making it difficult to encode something like a bootstrapping compiler.
Generating source code is not really supported -- a problem typically solved by
committing generated files to version control. Both tools also support build
clusters, which build code once and share the resulting objects to everyone
without recompiling locally -- an essential feature at such scales.

\vspace{-1mm}
\section{Conclusions and Future Work\label{section-conclusions}}

We have demonstrated that Make really is unsuitable for large complex build
systems, regardless of whether used recursively or non-recursively. Using Shake
we have rewritten the GHC build system, producing the fifth and hopefully final
version. While all previous versions have started simple and gained complexity
as they progressed, this version is different. Developing the abstractions in
\S\ref{sec:abstractions} took many months of discussion and refinement. Once the
fundamental concepts were in place, the rest was ``just'' coding and reverse
engineering the existing system. The result is faster, more maintainable and
more correct. There are three major tasks remaining for future work:

\begin{itemize}
\item While we have demonstrated that our approach works, we have not yet
implemented all features of the build system, and hope to do so over the next
few months. Once complete, we expect it to quickly become the only supported
method of building GHC.

\item Our abstractions from \S\ref{sec:abstractions} were designed to allow
tracking provenance of command line arguments -- mapping each flag to the
location of the expression that generated it. This feature will rely on the
\emph{implicit locations} feature of the latest GHC.

% using the new \emph{implicit locations}
% feature\footnote{\scriptsize\urlstyle{sf}\url{https://ghc.haskell.org/trac/ghc/wiki/ExplicitCallStack/ImplicitLocations}$\!\!\!$}
% in GHC

\item While faster than the old system, the build is still slower than we would
like. The zero build time could be reduced by switching to a faster serialisation
library. The critical path of a full rebuild takes over seven minutes, limiting
the gains available from additional processors. We hope to break this critical path
by refactoring the build system, which is now a feasible task.
\end{itemize}


% \acks
% Thanks to Standard Chartered, where Shake was initially developed.

% \vspace{2mm}
% \noindent
% \footnotesize{Neil Mitchell is employed by Standard Chartered Bank. This paper has been created in a personal capacity and Standard Chartered Bank does not accept liability for its content. Views expressed in this paper do not necessarily represent the views of Standard Chartered Bank.}

\bibliographystyle{plainnat}
\balance
\bibliography

\end{document}
